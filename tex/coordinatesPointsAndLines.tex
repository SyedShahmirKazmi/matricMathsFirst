\باب{جزو ضرب اور مضرب}


\ابتدا{مثال}
عدم مساوات کو حل کریں \((x-2)(x-4)<0\)

پہلا طریقہ : \(y=(x-2)(x-4)\) کی ترسیم کریں۔ یہ ترسیم x محور کو \(x=2\) اور \(x=4\) پہ کاٹے گی۔ اب جب کہ \(x^2\) کا عددی سر مثبت ہے، قطع مکافی اوپر کو جائے گا، جیسا کہ شکل 5۔5 میں دکھایا گیا ہے۔

آپ کو  x کی وہ قیمت معلوم کرنی ہے جہاں \(y<0\) ہو سکے۔
اس ترسیم سے آپ دیکھ سکتے ہیں کہ یہ تب ہو گا جب x  2 اور 4 کے درمیان ہو گا، یعنی \(x>2\) اور \(x<4\)۔
ہم جانتے ہیں کہ \(x>2\)  کا مطلب بھی وہی ہے جو \(2<x\) کا ہے، لہٰذا ہم اسے \(2<x<4\) لکھ سکتے ہیں۔ اس کا مطلب یہی ہو گا کہ x 2 سے بڑا اور 4 سے چھوٹا ہے۔
جب آپ \(r<x\) اور \(x<s\) قسم کی عدم مساوات کو \(r<x<s\) کے انداز میں لکھتے ہیں تو اس کا لازمی مطلب یہ ہوتا ہے کہ\(r<s\) ورنہ\(7<x<3\) لکھنا تو بالکل ہی غلط ہے؛ بھلا ایسا کیسے ہو سکتا ہے کہ  \(x\)  سات  سے بڑا بھی ہو اور تین سے چھوٹا بھی!
\انتہا{مثال}

\kshortcut

