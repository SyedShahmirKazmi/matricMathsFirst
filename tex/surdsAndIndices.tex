\باب{تفرق}
\حصہ{اعداد کی مختلف اقسام}
پہلے پہل اعداد کو صرف گنتی کے لیے استعمال کیا جاتا تھا اور \(1,2,3,\dotsc\) یہ کافی تھا ، انکو قدرتی اعداد یا مثبت اعداد کہا جاتا ہے
تب یہ بات آشکار ہوئی کہ اعداد کو معیشت اور پیمائش کے لیے بھی استعمال کیا جا سکتا ہے۔ اس مقصد کے لیے کسر بھی درکار تھے، کسر اور عدد صحیح مشترکہ طور پہ معقول اعداد بناتے ہیں۔ یہ وہ اعداد ہیں جوکہ \(\frac{p}{q}\) کی صورت میں لکھے جا سکتے ہیں ، بشرطیکہ p    اورq  صحیح اعداد ہوں اور  q   صفر نا ہو۔

یونانیوں کی باقی دریافتوں میں ایک یہ بھی بہت اہم  دریافت تھی کہ کئی ایسے اعداد تھے جن کا ہم اوپر بتاۓ گۓ طریقے سے اظہار نہیں کع سکتے تھے۔ ایساے اعداد کو نا معقول اعداد کہا جاتا ہے، پہلا نامعقول عدد  \(\sqrt{2}\) تھا۔ جو کہ 1 اکائ لمبائ کے مربع کے وتر کی لمبای ہے اور فیثا غورث کے قانون سے اسے معلوم کیا گیا۔ وہ اصول جسکی بنیاد پر یونانیوں نے یہ ثابت کیا کے \(\sqrt{2}\) کسر کی شکل میں نہیں لکھا جا سکتا اسی اصول کی بناء پر ہم  یہ کہہ سکتے ہیں کہ کسی بھی مثبت عدد کی  جزر، کعبی جزر اور دیگر یا تو صحیح عدد ہوں گی یا نا معقول۔ کئی دیگر نا معقول اعداد بھی ہمارے جانے پہچانے ہیں جن میں  \(\pi\) زبان زد عام یے۔

معقول اور نا معقول اعداد مجموعی طور پر حقیقی اعداد کہلاتے ہیں، صحیح اعداد،معقول و نا معقول اعداد اور حقیقی اعداد سب مثبت، منفی یا صفر ہو سکتے ہیں۔

جب معقول اعداد کو اعشاریہ میں لکھا جاتا ہے تو اسکے ہندسے یہ تو متناہی ہوتے ہیں یہ خود کو بالترتیب دہرانا شروع کر دیتے ہیں مثال کے طور پر؛

\[\frac{7}{10}=0.7,\quad \frac{7}{11}=0.6363\dotsc,\quad\frac{7}{12}=0.5833\dotsc,\quad\frac{7}{13}=0.538 461 538 461 53\dotsc\]
\[\frac{7}{14}=0.5,\quad\frac{7}{15}=0.466\dotsc,\quad\frac{7}{16}=0.4375,\quad\frac{7}{17}=0.411 764 705 882 352941176\dotsc\]

اس کا الٹ بھی ممکن ہے۔ اگر ایک اعشاری عدد ختم ہو جاۓ یا خود کو ہی دہراۓ تو یہ ایک معقول عدد ہوگا۔ لہٰذہ اگر ایک نا معقول عدد کو اعشاریہ کی صورت میں لکھا جاۓ ہندسوں کی ترتیب کبھی خود کو نہیں دہراتی چاہے آپ جتنا لمبا حساب کتاب کر لیں۔

%page 18

غیر معقول جزر کی خصوصیات

جب آپ \(\sqrt{2}\)،\(\sqrt{8}\) یا    \(\sqrt{12}\)    ایسے کلیہ کو آپ حساب کتاب کے آلہ کی مدد سے حل کر کے اعشاریہ کی شکل میں ضرور لکھتے ہوں گے۔ یقینی طور آپ نے کبھی لکھا ہوگا۔

\(\sqrt{2}=1.414\dotsc\)
 یا  
\(\sqrt{2}=1.414\)
جو اعشاریہ کے بعد تین ہندسوں
\(\sqrt{2}\approx 1.414\)
تک درست ہے
 
یہ کلیہ \(\sqrt{2}=1.414\) غلط کیوں ہے؟

  \(\sqrt{2}\)،\(\sqrt[3]{9}\) ایسے کلیوں کو ہی
غیر معقول جزر کہتے ہیں۔

اس ضیمہ میں ہم ایسے ہی غیر معقول جزر حل کریں گے۔ یہ بات ذہن نشین کر لیں کہ \(\sqrt{x}\) کا مطلب ہمیشہ x کی  مثبت جزر ہے یا صفر اگر x صفر ہو تو۔

غیر معقول جزر کی خصوصیات جو یہاں استعمال ہوں گی وہ درج ذیل ہیں۔


\(\sqrt{xy}=\sqrt{x}\times\sqrt{y}\) اور \(\sqrt{\frac{x}{y}}=\frac{\sqrt{x}}{\sqrt{y}}\)

آپ یہ دیکھ سکتے ہیں کہ \((\sqrt{x}\times\sqrt{y})\times(\sqrt{x}\times\sqrt{y})=(\sqrt{x}\times\sqrt{x})\times(\sqrt{y}\times\sqrt{y})=x\times y=xy\) اور \(\sqrt{x}\times\sqrt{y}\) مثبت ہے، یہ  
 \(xy\) 
کی جزر ہے۔ اسی لیے \(\sqrt{xy}=\sqrt{x}\times\sqrt{y}\)۔ یہی وجہ آپ کو اس بات پر بھی قائل کرے گی کہ\(\sqrt{\frac{x}{y}}=\frac{\sqrt{x}}{\sqrt{y}}\)

درج ذیل مثال سے انہی خصوصیات کو واضع کرتی ہے۔
\[\sqrt{8}=\sqrt{4\times 2}=\sqrt{4}\times\sqrt{2}=2\sqrt{2};\quad\sqrt{12}=\sqrt{4\times 3}=\sqrt{4}\times\sqrt{3}=2\sqrt{3};\]
\[\sqrt{18}\times \sqrt{2}=\sqrt{18\times 2}=\sqrt{36}=6;\quad \frac{\sqrt{27}}{\sqrt{3}}=\sqrt{\frac{27}{3}}=\sqrt{9}=3.\]

آپکو اپنے طور پر بھی اوپر دیے گۓ حساب کتاب کو دہرانا چاہیے۔



مثال 2.2.1

حل کریں 
ا \[\sqrt{28}+\sqrt{63}\]
ب \[\sqrt{5}\times \sqrt{10}\]

حل کرنے کے دو طریقے ہیں اور دونوں حصوں کو مختلف طریقوں سے حل کیا گیا ہے۔
\[\sqrt{28}+\sqrt{63}=(\sqrt{4}\times\sqrt{7})+(\sqrt{9}\times\sqrt{7})=2\sqrt{7}+3\sqrt{7}=5\sqrt{7}\]

 طریقہ 1 \(\sqrt{5}\times\sqrt{10}=\sqrt{5\times10}=\sqrt{50}=\sqrt{25\times 2}=5\sqrt{2}\)
طریقہ 2 \(\sqrt{5}\times\sqrt{10}=\sqrt{5}\times(\sqrt{5}\times\sqrt{2})=(\sqrt{5}\times\sqrt{5})\times\sqrt{2}=5\sqrt{2}\)
  بعغ دفعہ کسر میں نسب نما سے جزر کو ختم کرنا ہمارے لیے مفید ہوتا ہے، جیسا کہ \(\frac{1}{\sqrt{2}}\) آپ اسکہ با آسانی حل کر سکتے ہیں اگر آپ \(\sqrt{2}\) سے ضرب بھی دیں اور تفریق بھی کریں۔\(\sqrt{2}:\frac{1\times\sqrt{2}}{\sqrt{2}\times\sqrt{2}}=\frac{\sqrt{2}}{2}\)



% page number 75



%page 76

\[ y=x^{2}\]
\[ (0.4 , 0.16) \quad (0.7 , 0.49)\]

\( \delta\)
\(\delta x\)
\(\delta y\)
\(x\)
\(y\)




\[ \]

%page number 77




% maths book in urdu


یہ سبق کسی بھی  ترسیم پر موجود نقطے کے ڈھلاؤ یا خط مماس معلوم کرنے کے بارے میں ہے۔  جب آپ یہ سبق مکمل کر لیں گے ، آپ کو عبور حاصل ہوگا کہ:
 
 
آپ ایک سمتی مقدار پر ایک نقطہ پہ  ڈھلاؤ معلوم کرنے کے لئے ایک کلیہ کا حساب لگائیں, اس کی مساوات بنائیں
 مربعی اور دیگر قسم کے خم پہ ایک نقطہ پر عین مطابق ڈھلاؤ کا حساب لگائیں 
 

اس سبق کو دو حصوں میں تقسیم کیا گیا ہے۔ پھلے حصے میں ضیم 6.1 تا 6.5 میں آپ تجربے کی بنیاد پر نتا ئج اخذ کرتے ہوۓ خط مماس سے ترسیم تک کے مسائل حل کریں گے۔ دوسرے حصے میں ضیمہ 6.6 تا 6.7   تجربے کی بنیاد اخذ پر نتا ئج کو  ثابت کریں گے۔
آپ اگر چاہیں تو سبق کے دوسرے حصے کو نظرانداذ کر سکتے ہیں لیکن آپ کو چاھۓ کہ سبق کے اختتام پہ موجود مشق کو حل کریں۔




خط مماس کا ڈھلاؤ معلوم کرنا؛

ایک سادہ سے خم کے بارے میں سوچیں جیسے کے  \( y=x^{2}\) کی ترسیم۔ جیسے جیسے  آپ کی نظر (x-axis)کی سمت بڑھتی ہے ، کیا آپ بیان کر سکتے ھیں ، ریاضی کی زبان میں، خم کی سمت کس طرہ سے تبدیل ھوتی ہے۔

جیسے ایک سیدھی لکیر کا ایک عددی ڈھلاءو ہے، لہٰذہ کوئ بھی خم ، بشرطیکہ وہ کافی حد تک (سموتھ)ہو ایک ڈھلوان یا ڈھلاءو رکھتا ہے جو کہ کسی بھی ایک نقتے پے ماپا جا سکتا ہے۔ فرق صرف اتنا ھے کہ خم کے لیے ڈھلاءو کی سمت بھی بدلتی ھے جیسے جیسے آپ اسکے ساتھ چلتے ھیں۔ ریاضی دان اس ڈھلاءو کی مدد سے خم کی سمت کا تعین کرتے ہیں۔


سبق نمبر 1 میں آپ سیکھ چکے ھیں کے اگر آپکے پاس ایک سیدھی لکیر کے دو نقتوں کے محدد دستیاب ہون تو آپ کیسے اسکا ڈھلاءو معلوم کر سکتے ہیں۔ آپ اس طریقے کو براہ راست خم پے استعمال نہیں کر سکتے کیوںکہ وہ ایک سیدھی لکیر نہیں ہے۔ آپ اس خط مماس کا ڈھلاءو معلوم کرتے ھیں جو کہ خم کے کے کسی بہی دو نقتوں کی مدد سے بنایا جاۓ گا۔(جیسا کہ آپ تصویر 6.1 میں دیکھ رھے ھیں) کہ ایک نقتے پر خط مماس اور ڈھلاءو کی ڈھلوان برابر ہے۔تاھم یھاں ایک نیا مسعلہ کھڑا ہو گیا ہے، وہ یہ کہ آپ ایک لکیر کا ڈھلاءو صرف تب ہی معلوم کر سکتے ہیں جب آپ کو اسکے دو نقتوں کے محدد  پتا ہوں۔

تصویر 6.2 میں ہم آہنگ لکیریں (سیدھی لکیریں جو خم کے دو نقتوں میں سے گزریں) دکھائ گئ ہیں جو خط مماس کے قریب تر ہوتی جا رہی ہیں، لہٰذہ بہتر یہی ہے کہ ہم ان ھم آھنگ لکیروں کی ڈھلوان معلوم کرنے سے ابتدا کریں، کیونکہ اس طریقے کو پہلے سبق میں سیکھ چکے ھیں۔


%page 76


مثال 6.1.1

ایک ھم آھنگ لکیر کی ڈھلوان اور مساوات معلوم کریں جو کہ \( y=x^{2}\) کے خم کے دو نقتوں کو جورتی ہے ، ان دو نقتون کے محدد ہیں (0.4،0.16) اور(0.7،0.16)

ضیمہ 1.3 میں ہم آہنگ لکیر کی ڈھلوان معلوم کرنے کے نسخہ کے مطابق؛
\[ \frac{0.49-0.16}{0.7-0.4} = \frac{0.33}{0.3} = 1.1\]

ضیمہ 1.5 میں ہم آہنگ لکیر کی مساوات معلوم کرنے کے نسخہ کے مطابق؛
\[ y-0.16 = 1.1(x-0.4), \]
جو کہ؛
\[ y=1.1x-0.28\]


figure 6.3 caption

یہاں مددگار ہوگا کہ ہم کچھ نئ علامات سیکھیں، بڑھوتری کے لئے ہم یونانی علامت \( \delta\)(ڈیلٹا) کا استعمال کرتے ہیں۔لہٰذہ \(x\)  میں بڑھوتری کو        \(\delta x\)    اور          \(y\)  میں بڑھوتری کو          \(\delta y\) سے ظاہر کرتے ہیں۔ ان مقداروں کو ضیمہ 1.3 میں (x-step) اور (y-step) کہا گیا ہے۔ لہٰذہ مثال 6.1.1 میں ھم آھنگ لکیر کے ایک سرے سے دوسرے (x-step)    \( 0.7-0.4=0.3\) جبکہ (y-step) \(0.49-0.16=0.33\) ہے۔لہٰذہ ہم کہ سکتے ھیں کہ؛
\[ \delta x = 0.3 ,\quad \delta y =0.33\]

اس علامت نویسی سے آپ کسی ھم آھنگ لکیر کی ڈھلوان کو \( \frac{\delta y}{\delta x}\) سے ظاہر کر سکتے ہیں۔ 

ایک طبقہ  \( \delta\) کی بجاۓ                 \( \Delta \)             استمعال کرتا ہے، دونوں ہی صحیح ہیں۔   
اس بات کا خیال رکھیں کہ اس \(frac{\delta y}{\delta x}\) میں آپ دونوں \(\delta\) کو آپس میں کاٹ نہیں سکتے کیوںکہ یہ اعداد نہیں ہیں۔
اب جبکہ ہم علامت کو استعمال کر رھے ہیں تو آپ عادت بنا لیں اسکو باڑھوتری کی شرح کہنے کی ، اسطرح آپ اسکو الجبرا کی عام علامت نا سمجھیں۔ یاد رکھیں کہ \(\delta x\) یا\(\delta y\)  منفی بھی یو سکتے ہیں،اور ایسی صورت میں (x-step)اور(y-step) کم ھونگے۔
\(\delta y\))()

اگر ہم اس طریقے کو بروے کار لاتے ہوے ھم آھنگ لکیروں کے ڈھلاءو کو معلوم کریں تو مثال 6.1.1 کچھ ایسی دکھے گی؛

\[ \frac{\delta y}{\delta x} = \frac{0.49-0.16}{0.7-0.4}=\frac{0.33}{0.3}=1.1 \]


مثال 6.1.2

خم \( y=x^{2}\) کے دو نقتوں کو جوڑنے والی ھم آھنگ لکیر کا ڈھلاءو معلوم کریں ۔جسکے x محدد 0.4 اور0.41 ہیں۔
سب سے پہلے آپکو ان دونوں نقتوں پہ    y
محدد معلوم کرنا ھوگا  جوکہ \( 0.4^{2}=0.16\)  اور \( 0.41^{2} =0.1681\) ہیں۔
     
     
% page 77


 اس مثال کو بھی مثال 6.1.1 کی طرہ حل کریں گے۔\( \delta x = 0.41-0.4=0.01\)  اور   \(\delta y =0.1681-0.16=0.0081\)
تاہم ھم آھنگ لکیر کا ڈھلاءو ھے؛
\[ \frac{\delta y}{\delta x} =\frac{0.0081}{0.01}=0.81\]
  
مثال 6.4 میں  ھم آھنگ لکیر اور  xمحدد 0.4 پہ موجود خط مماس کو الگ پہچاننا  مشکل ھے ، (((((یہاں ہمیں ایک طریقہ ملتا ھے کہ ہم  xمحدد 6.4 پہ خط مماس کا ڈھلاءو کسطرہ معلوم کر سکتے ھیں۔)))))
مثال 6.1.3 میں یہ نقطے مزید قریب آگۓ ہیں۔

مثال 6.1.3
خم \( y=x^{2}\) کے دو نقتوں کو جوڑنے والی ھم آھنگ لکیر کا ڈھلاءو معلوم کریں ۔جسکے x محدد 0.4 اور0.40001 ہیں۔
دونوں نقتوں کے محدد  \( ( 0.4,0.4^{2} ) \)  اور            \( (0.40001 , 0.40001^{2}) \)           ہیں 

\( \delta x = 0.40001-0.4=0.00001\)  اور   \(\delta y =0.40001^{2}-0.4^{2}=0.0000080001\)
لہٰذہ ھم آھنگ لکیر کا ڈھلاءو ہے؛
      \(\frac{\delta y}{\delta x}=\frac{0.0000080001}{0.00001}=0.80001\)          

یہ نتیجہ چونکہ 0.8 کے بہت قریب ہے لہٰذہ ایسا نظر آ رہا ہے کہ\( y=x^{2} \) کے خم پہ xمحدد  0.4 پہ موجود     خط مماس کا ڈھلاءو  0.8 ہے ۔ لیکن اس سے یہ ثابت نہیں ہوتا کیونکہ آپ ابھی تک دو نقتوں کو ملانے والی خط مماس کی مساوات معلوم کر رہے ہیں،قطعی نظر اسکے کہ یہ نقطے کتنے قریب ہیں۔


%                             EXERCISE  6A

سوالنمبر 2 اور 3 میں سوالوں کو مختلف حصوں میں تقسیم کیا جا سکتا ہے،لہٰذہ طلباء کی جماعت اکٹھے کام کرتے ہوۓ ان تمام سوالوں کے جوابات حاصل کر لیں گے،اور پھر ان جوابات کو جمع کیا جا سکتا ھے،

question 1
سیدھی لکیر کی ایک مساوات بنائیں جو \( y=x^{2} \)کے خم پہ دو نقتوں کو ملاۓ جنکی x کی قیمت 1 اور 2 ہو۔

question 2
اس سوال کے ہر حصے میں  \( y=x^{2} \)کے خم پہ  درج ذیل  x محدد  کے  دو نقتوں  سے بننے والی ھم آھنگ لکیروں کا ڈھلاءو معلوم کریں ۔
   
1اور    1.001
1 اور    0.9999    
2 اور   0.002
2 اور 1.999
3 اور 3.000001
3 اور 2.99999

question 3

اس سوال کے ہر حصے میں، \( y=x^{2} \)کے خم پہ درج ذیل نقتے اور اسکے قریبی نقتے کی مدد سے بنی    ھم آھنگ لکیر کا ڈھلاءو معلوم کریں۔ بتاۓ گۓ نقتے اور اسکے قریبی نقتے کے مابین فاصلے کو تبدیل کر کے اسی عمل کو دہرائیں، اس بات کا خیال رکھیں کہ کچھ نقتے بتاۓ گۓ نقتے کی بائیں طرف بھی ہوں۔
(-1،1)  
(-2،4)
(10،100)
  

question 4
سوالنمبر 2 اور 3 سے حاصل شدہ تجربے کی بنیاد \( y=x^{2} \)کے خم پے موجود کسی بھی نقتے پے خط مماس کے ڈھلاءو کا انداذہ کریں۔

question 5
part a
سوالنمبر 2 سے 4 تک استعمال شدہ طریقے کی مدد سے \( y=x^{2} +1 \)اور \( y=x^{2} -2 \)کے خم پے موجود کسی بھی نقطے پہ بنی خط مماس  کا ڈھلاءو معلوم کریں۔
part b
حصہ (ا) میں حاصل شدہ تجربے کی بنیاد پے\( y=x^{2} +c \) ، جبکہ c ایک حقیقی عدد ھے،   کے خم پہ بنی کسی بھی خط مماس کے ڈھلاءو کو معلوم کرنے کا ایک عالمگیر اصول وضع کریں 

خط مماس کا ڈھلاءو جو کہ  \( y=x^{2} +c \)کے خم پے بنی ھو۔


اگر آپ مشق 6A کے نتاءج کو جمع کریں تو آپکو یہ شبہ گزرے گا کہ کسی بھی نقتے پر\( y=x^{2}\) کے خم پے بنی خط مماس کا ڈھلاءو x محدد کا دو گناہ ھے، لب لباب یہ یے کہ \( y=x^{2}\)کے خم کے ڈھلاءو کا کلیہ 2x     ہے  

مثال کے طور پر ،  \( y=x^{2}\) کے  خم پے  ایک نقتے (3،9-) کا ڈھلاءو \( 2 \times (-3) =-6 \) ہے،اسکا مطلب یہ ھوا کہ اس نقتے پہ بنی خط مماس کا ڈھلاءو -6 ہوگا اور یہ نقتہ    ( 3،9-)  سے گزرے گی۔
اس خط مماس کی مساوات معلوم کرنے کے لیے آپ ضیمہ 1.5 کا سہارا لے سکتے ہیں۔ لکیر کی مساوات ہو گی؛
\[y-9=-6(x-(3)),\]
جو کہ ؛
\(y-9 =6x-18\)
یا
\(y=-6x-9\)
ہے۔

غور کریں کہ ڈھلاءو کا کلیہ  خم \( y=x^{2} +c \) پر بھی لاگو ھو رہا ہے ، جبکہ c مستقل ہے۔ xپے بھی ڈھلاءو 2x ہی ہے۔ اسکی وجہ سادہ سی ہے کہ  \( y=x^{2} +c \) کا خم بھی  \( y=x^{2}  \)  کے خم جیسا ہی ھے بس صرف   y
محور کی سمت تھوڑا منتقل ہو گیا ہے۔

ایک ثانیے کے لیے یہ مان لیں کہ ان نتائج کو ثابت کیا جا سکتا ہے۔آپکو ثبوت مل جاۓ گا ضیمہ 6.6 میں۔

خم کے کسی نقتے پر عمودی لکیر۔

وہ لکیر جو خم اور خط مماس کے باہمی ملاپ کے نقتے سے کچھ اس طرح گزرے کہ خط مماس کے ساتھ  \( 90^0\) زاویہ بناۓ  اس نقتے پر خم کی عمودی لکیر کہلاتی ہے۔

\( 90^0\)



 %page 79

اگر آپکو کسی ایک نقتے پر خط مماس کا ڈھلاءو  معلوم ہےتو آپ ضیمہ 1.9 کے نتیجے سے  عمودی خط کا ڈھلاءو معلوم کر سکتے ہیں۔ اگر خط مماس کا ڈھلاءو  m
یے تو عمودی لکیر کا ڈھلاءو \(- \frac{1}{m}\) ہوگا، بشرطیکہ \(m \ne 0\)

مثال 6.3.1
  
 \( y=x^{2} \)   
 کے خم پہ  عمودی لکیر کی ایک مساوات معلوم کریں جبکہ اس نقتے پہ ؛\(x=-3\) اور \(x=0\)  ہے۔

ضیمہ 6.2 میں ہم نے (3،9-) پے خط مماس  کا ڈھلاءو معلوم کیا تھا جو کہ -6  تھا۔ اسی طرح عمودی خط کا ڈھلاءو \(-\frac{1}{-6} =\frac{1}{6}\) اور یہ بھی (3،9-) سے ہی گزرتا ہے۔ لہٰذہ عمودی خط کی مساوات \( y-9=\frac{1}{6}(x-(-3)),\) جسکی سادہ شکل
\( 6y=x+57\)
ہے۔

نقتے \( (0,0)\)  پر خط مماس کا ڈھلاءو  0 ہے۔ لہٰذہ  خط مماس x محور کے مساوی بڑھتی رہے گی۔ اسی وجہ سے عمودی خط y محور کے مساوی بڑہتی رہے گی۔ اسکی مساوات کچھ یوں ہے۔\(x= کچھ بھی \)   ۔ جب عمودی خط \( (0,0)\)  سے گزرتی یے تو اسکی  مساوات ہوگی \(x= 0 \) 

اگر آپکے پاس ترسیم کاری کا آلہ موجود یے تو ، آپ\( y=x^{2}  \) کے خم کو، \( y=-6x-9  \) کے خط مماس کو، \( 6y=x+57  \) کے عبوری خط کا مشاہدہ کریں، آپ نتائج سے حیران ضرور ہوں گے۔

آپکو یہ بات سمجھنی ھو گی کہ اگر آپ ایک ھی نقتے پر خم، خط مماس اور خط عمودی کشید کریں تو خط عمودی صرف اسی صورت\(90^0\) زاویے پر ہوگا جب x   اور  y محور دونوں پہ پیمانہ ایک سا ہو، خط مماس پر کوئ فرق نہیں پڑے گا۔  

اس موقع پر آپکو سمجھنا ھو گا کہ آپکو اس نتیجے کو دیگر مساوات کے لئے بھی علمگیر کرنا ھوگا۔ ضیمہ 6.2 میں آپ نے دیکھا کہ \( y=x^{2} +c \) کے خم میں کسی بھی x پہ موجود خط مماس کا ڈھلاءو 2x کے برابر ہوگا۔

مشق 6B


سوالنمبر 9  سے 12 میں سوالوں کو مختلف حصوں میں تقسیم کیا جا سکتا ہے،لہٰذہ طلباء کی جماعت اکٹھے کام کرتے ہوۓ ان تمام سوالوں کے جوابات حاصل کر لیں گے،اور پھر ان جوابات کو جمع کیا جا سکتا ھے،

درج ذیل x  محدد کی مدد سے \( y=x^{2}\)  کے خم پے    بننے والے خط مماس کا ڈھلاءو معلوم کریں۔ 
 1
4
0
-2
-0.2
-3.5
p
2p

%question 2


درج ذیل x  محدد کی مدد سے \(2- y=x^{2}\)  کے خم پے    بننے والے خط مماس کا ڈھلاءو معلوم کریں۔ 
1
4
0
-2
-0.2
-3.5
p
2p


%question 3

خم  \(5+ y=x^{2} ۤۤ\) پہ ایک نقطے  p کا    y محدد  9 ہے۔ اسی نقطے پے خط مماس کے ڈھلاءو کی دو ممکن مقداریں معلوم کریں۔
%page 80
%question 4
 بتاۓ گۓ خم کے دیۓ گۓ x  یا   y محدد سے معرض وجود میں آۓ نقتے پر بنی خط مماس کی مساوات معلوم کریں۔
\( y=x^{2}\) جبکہ \( x=2\)
\( x=2 +2 \)  جبکہ \( x=-1\)
\( x=2 -2 \)   جبکہ    \( y=-1\)
 \( x=2 -2 \)  جبکہ    \( y=-2\)


%question 05

تاۓ گۓ خم کے دیۓ گۓ x  یا   y محدد سے معرض وجود میں آۓ نقتے پر بنے عمودی خط  کی مساوات معلوم کریں  
\( y=x^{2}\)        جبکہ      \(x=1\)
\( y=x^{2}+1\)   جبکہ       \(x=-2\)        
\( y=x^{2}+1\)     جبکہ      \(x=0\)
\( y=x^{2}+c\)     جبکہ       \(x=\sqrt{c}\)
 %question 6
خم \( y=x^{2}\) کے ایک نقتے P پرخط مماس کا ڈھلاءو 3 ہے۔ اس نقتے P پر عمودی خط کی مساوات بنائیں۔

%question 7

خم  \( y=x^{2}+1\) کے ایک نقتے P پر عمودی خط  کا ڈھلاءو-1 ہے۔ اس نقتے P پر  خط مماس  کی مساوات بنائیں۔
%question 8
خم \( y=x^{2}\) میں محدد \( (2,4) \) پہ بننے والا عمودی خط دوبارہ اس خم سے گزرتا ہے، اس نقتے کی نشاندہی کریں۔

%question 9
 سوال کے ہر حصے میں درج ذیل
  خموں کو بروۓ کار لاتے ہوۓ   \( y=2x^{2}\)، \( y=3x^{2}\)اور\( y=-x^{2}\)  بتاۓ گۓ x محدد سے بننے والے نقتوں سے معرض وجود میں آنے والی ھم آھنگ لکیر کا    ڈھلاءو معلوم کریں 
1       1.001
1    اور    0.99999
2    اور      2.002 
2      اور     1.999
3       اور    3.000001
3        اور   2.99999
%question 10
اس سوال کے ہر حصے میں، \( y=\frac{1}{2} x^{2}\) اور  \( y=\frac{1}{2} x^{2}+2\)   کے خم پہ درج ذیل                 x محدد سے بننے والے  نقتے اور اسکے قریبی نقتے کی مدد سے بنی    ھم آھنگ لکیر کا ڈھلاءو معلوم کریں۔ بتاۓ گۓ نقتے اور اسکے قریبی نقتے کے مابین فاصلے کو تبدیل کر کے اسی عمل کو دہرائیں، اس بات کا خیال رکھیں کہ کچھ نقتے بتاۓ گۓ نقتے کی بائیں طرف بھی ہوں۔
-1
-2
10



%question 11

سوالنمبر9 اور 10 سے حاصل شدہ تجربے کی بنیاد \( y=ax^{2} \) اور    \( y=ax^{2} +c \) ، جبکہ aایک حقیقی عددہے،  کے خم پے موجود کسی بھی نقتے پے خط مماس کے ڈھلاءو کا انداذہ کریں۔



%question 12

question 5
part a
سوالنمبر9 سے 11 تک استعمال شدہ طریقے کی مدد سے      \( y=x^{2} +3x\)اور\( y=x^{2} -2x\) کے خم پے موجود کسی بھی نقطے پہ بنی خط مماس  کا ڈھلاءو معلوم کرنے کا طریقہ وضع کریں۔
part b
حصہ (ا) میں حاصل شدہ تجربے کی بنیاد پے\( y=x^{2} +bx\) ، جبکہb ایک حقیقی عدد ھے،   کے خم پہ بنی کسی بھی خط مماس کے ڈھلاءو کو معلوم کرنے کا ایک عالمگیر اصول وضع کریں 

دو درجی ترسیم کے ڈھلاءو کا کلیہ؛
سبق 3 میں عام دو درجی ترسیم کا تزکرہ کیا گیا ،جسکی مساوات کچھ ایسی  \(y=ax^{2}+bx +c\) ہے، جنکہ ،a      b         اور   c  مستقل ہیں۔ ایسے خم کی خط مماس کے ڈھلاءو کے بارے میں آپکا کیا خیال ہے؟

مشق  6B کے سوالات کے حل سے آپکو اندازہ ہوا ہوگا \( y=ax^{2}\) کے ڈھلاءو کا کلیہ  2ax یے. اسکا مطلب، مثال کے طور پہ،        \(y=3x^{2}\)    کے خم کا ڈھلاءو   x کی ہر مقدار پہ تین گناہ ذیادہ ہو گا اگر اسکے مقابل خم \(y=x^{2}\) کا ہو۔ آپنے اس بات کا بھی مشاہدہ کیا ہوگا کہ \(y=x^{2}+bx\)     کے ڈھلاءو کا کلیہ     \(2x +b\)  یے، لب لباب تمام کلام کا یہ ہے کہ  \(y=x^{2} +4x\)  کے  ڈھلاءو کا کلیہ   \( 2x+4\)  ہے۔ جوکہ \(x^{2}\) اور \( 4x\) کے ڈھلاءو کے کلیوں کے جمع کے برابر ہے۔


%page 81
آپکو اس بات کا پہلے سے ہی علم ہے کہ    \(y=x^{2}\)   اور  \( y=x^{2}+c\)   دونوں کے ڈھلاءو کا کلیہ ایک ہی ہے،     c کی مقدار جتنی بھی ہو اس سے فرق نہیں پڑتا۔

لہذہ اس بات میں کوی بعید نہیں یے کہ؛

مساوات  \( y=ax^{2}+bx+c\)   کے خم کے  ڈھلاءو کا کلیہ       2ax +b    ہے،


یہ نتیجہ اس طرح بھی اہم ہے کہ ھم ایک ایسے تفاعل کا ڈھلاءو معلوم کر سکتے ہیں جو کئ حصوں پر مشتمل ہو، اور ایسا کرنے کے لیۓ ہم ان تمام حصون کے ڈھلاءو کو باہمی طور پے جمع کر دیتے ہیں۔
آپ ایک ایسے تفاعل کا ڈھلاءو بھی  معلوم کر سکتے ہیں جسکے ساتھ کوی مستقل عدد ضرب کھا رہا ہو، اور ایسا کرنے کے لیۓ ہمیں اس تفاعل کے ڈھلاءو کو بھی اسے عدد سے ضرب دینی ہو گی۔

ضیمہ  6.6 میں ہم اس بات کا مشاہدہ کریں گے کہ ہم ان نتائج کو ثابت کر سکتے ہیں، وقت ضائع نہ کرتے ہوۓ یہاں ان کے استعمال کی چند مثالیں دی گئ ہیں ، لیکن اس سے قبل ہمیں علامت نویسی کو دیکھنا ہوگا۔


مان لیں کے ایک خم کی مساوات \(y=f(x) \)  ہے، اسکے ڈھلاءو کا کلیہ \( f\prime (x)\) ہوگا، اور اسکو  f    ڈیش  x پڑھا جاۓ گا۔

کسی بھی خم کے خط مماس کے ڈھلاءو کو معلوم کرنا تفریق کاری کہلاتا یے، اور جب ہم اس عمل کو انجام دے رہے ہوتے ہیں تو دراصل ہم تفریق کاری کر رہے ہوتے ہیں۔


جیسے f(2)  سے مراد وہ مقدار ہے جب \( x=2\) ہو، اسی طرح  \( f\prime (2)\) اس ڈھلاءو کے لیے استعمال ہوتا ہے جب \( y=f(x)\) اور \(x=2\) ہو۔ لہٰذہ (ڈیش) جو \( f \prime (x)\) پے بنی ہے ہمیں تفریق کا بتاتی ہے، اس علامت کو دیکھتے ہی آپ جس  x پہ ڈھلاءو معلوم کرنا چاہتے اسکے x کے متبادل سے x کو بدل دیتے ہیں۔

مقدار \( f\prime (2)\) کو تفاعل \( f(x)\) کا تفرق کہا جاۓ گا جب \(x=2\) ہوگا۔

  
پس جب بھی ہمیں مساوات   y=f(x) کے خم کا   \(x=2\) پے ڈھلاءو  معلوم کرنا ہو ، ہم \(f\prime (x)\) معلوم کریں گے اور پھر x کو اسکے متبادل مقدار یعنی یہاں 2 سے بدل دیں گے، نتیجے کے طور پر ہمیں \( f\prime (2)\) ملے گا۔


%example 6.4.1

مثال 6.4.1

مساوات \( 3x^{2}-2x+5\) کو تفریق کریں۔
خم \( 3x^{2}-2x+5\) کے خط مماس اور عمودی خط کے ڈھلاءو کی مساوات بنائیں، جبکہاس نقتے پر \(x=1\) ہو۔

مان لیں کہ \(f(x)=3x^{2}-2x+5\)، اس تفاعل کے مطابق    a=3         ،      b=-2        اور     c=5
ہے ۔ اس تفاعل کا اگر تفرق معلوم کریں،\( f\prime(x) =2\times3\times x-2 = 5x-2\)

وہ نقتہ جس پہ x=1 اسکا  y محدد برابر ہوگا   3-2+5=6

جب x=1 ہوگا تو خط مماس کا ڈھلاءو ہوگا،\(f\prime (1)=6\times 1-2-4\)

اسی لیے خط مماس کی مساوات کچھ یوں بنے گی،         \( y-6=4(x-1)\)    یا\(y=4x+2\)

عمودی خط، خط مماس کے ساتھ عمودی ہوتا ہے اس لیے اسکا ڈھلاءو   \( \frac{-1}{4}\)  ہوگا، اسی لیے عمودی خط کے ڈھلاءو کی مساوات      \(y-6=\frac{-1}{4}(x-1)\)  ہوگی جسکی سادہ شکل کچھ یوں \(x+4y=25\) ہے۔


%page 82
